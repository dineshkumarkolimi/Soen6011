%! Author = DineshkumarKolimi
%! Date = 2022-08-01

% Preamble
\usepackage{url}
\documentclass[11pt]{article}

% Packages
\usepackage[utf8]{inputenc}
\usepackage{hyperref}
\title{SOEN6011}
\author{Dineshkumar Babu Kolimi - 40094976}
\date{\today}

% Document
\begin{document}
    \maketitle
    \section{Introduction:}
    	This document shows the basic understanding of the function \sinh{x}.
    \section{Function:}
    	The hyperbolic sine function is
    \begin{equation}
        \sinh(x) &= \frac{e^x - e^{-x}}{2}
    \end{equation}
    \begin{equation}
        e &= \lim_{x \to \infty}(1+\frac{1}{x})^x
    \end{equation}
   	 e = 2.71828182 (approximately)
    \section{Domain and Range:}
	The domain of hyperbolic sine function is (-\infty,+\infty).
	\\The range of hyberbolic sine function is [-1,1].
  

    \section{Characteristics:}
    \begin{itemize}
        \item Sinh (-x) = -sinh x
        \item d/dx sinh (x) = cosh x
        \item For large positive x Sinh x = Cosh x
        \item For large negative x Sinh x = -Cosh x
        \item Sinh 2x = 2 sinh x cosh x
        \item Sinh x = – i sin(ix)
        
    \end{itemize}
    \section{References:}
    \begin{thebibliography}{9}
        \bibitem{}
        \url{http://www.mathcentre.ac.uk/resources/workbooks/mathcentre}

        \bibitem{}
        \url{hyperbolicfunctions.pdf}

        \bibitem{}
        \url{https://www.analyzemath.com/DomainRange/domain_range_functions.html}

    \end{thebibliography}

\end{document}